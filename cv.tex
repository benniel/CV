%%%%%%%%%%%%%%%%%%%%%%%%%%%%%%%%%%%%%%%%%%%%%%%%%%%%%%%%%%%%%%%%%%%%%%%%
%%%%%%%%%%%%%%%%%%%%%% Simple LaTeX CV Template %%%%%%%%%%%%%%%%%%%%%%%%
%%%%%%%%%%%%%%%%%%%%%%%%%%%%%%%%%%%%%%%%%%%%%%%%%%%%%%%%%%%%%%%%%%%%%%%%

%%%%%%%%%%%%%%%%%%%%%%%%%%%%%%%%%%%%%%%%%%%%%%%%%%%%%%%%%%%%%%%%%%%%%%%%
%% NOTE: If you find that it says                                     %%
%%                                                                    %%
%%                           1 of ??                                  %%
%%                                                                    %%
%% at the bottom of your first page, this means that the AUX file     %%
%% was not available when you ran LaTeX on this source. Simply RERUN  %%
%% LaTeX to get the ``??'' replaced with the number of the last page  %%
%% of the document. The AUX file will be generated on the first run   %%
%% of LaTeX and used on the second run to fill in all of the          %%
%% references.                                                        %%
%%%%%%%%%%%%%%%%%%%%%%%%%%%%%%%%%%%%%%%%%%%%%%%%%%%%%%%%%%%%%%%%%%%%%%%%

%%%%%%%%%%%%%%%%%%%%%%%%%%%% Document Setup %%%%%%%%%%%%%%%%%%%%%%%%%%%%

% Don't like 10pt? Try 11pt or 12pt
\documentclass[10pt]{article}

% This is a helpful package that puts math inside length specifications
\usepackage{calc}


% Simpler bibsection for CV sections
% (thanks to natbib for inspiration)
\makeatletter
\newlength{\bibhang}
\setlength{\bibhang}{1em}
\newlength{\bibsep}
 {\@listi \global\bibsep\itemsep \global\advance\bibsep by\parsep}
\newenvironment{bibsection}%
        {\vspace{-\baselineskip}\begin{list}{}{%
       \setlength{\leftmargin}{\bibhang}%
       \setlength{\itemindent}{-\leftmargin}%
       \setlength{\itemsep}{\bibsep}%
       \setlength{\parsep}{\z@}%
        \setlength{\partopsep}{0pt}%
        \setlength{\topsep}{0pt}}}
        {\end{list}\vspace{-.6\baselineskip}}
\makeatother

% Layout: Puts the section titles on left side of page
\reversemarginpar

%
%         PAPER SIZE, PAGE NUMBER, AND DOCUMENT LAYOUT NOTES:
%
% The next \usepackage line changes the layout for CV style section
% headings as marginal notes. It also sets up the paper size as either
% letter or A4. By default, letter was used. If A4 paper is desired,
% comment out the letterpaper lines and uncomment the a4paper lines.
%
% As you can see, the margin widths and section title widths can be
% easily adjusted.
%
% ALSO: Notice that the includefoot option can be commented OUT in order
% to put the PAGE NUMBER *IN* the bottom margin. This will make the
% effective text area larger.
%
% IF YOU WISH TO REMOVE THE ``of LASTPAGE'' next to each page number,
% see the note about the +LP and -LP lines below. Comment out the +LP
% and uncomment the -LP.
%
% IF YOU WISH TO REMOVE PAGE NUMBERS, be sure that the includefoot line
% is uncommented and ALSO uncomment the \pagestyle{empty} a few lines
% below.
%

%% Use these lines for letter-sized paper
\usepackage[paper=letterpaper,
            %includefoot, % Uncomment to put page number above margin
            marginparwidth=1.2in,     % Length of section titles
            marginparsep=.05in,       % Space between titles and text
            margin=1in,               % 1 inch margins
            includemp]{geometry}

%% Use these lines for A4-sized paper
%\usepackage[paper=a4paper,
%            %includefoot, % Uncomment to put page number above margin
%            marginparwidth=30.5mm,    % Length of section titles
%            marginparsep=1.5mm,       % Space between titles and text
%            margin=25mm,              % 25mm margins
%            includemp]{geometry}

%% More layout: Get rid of indenting throughout entire document
\setlength{\parindent}{0in}

%% This gives us fun enumeration environments. compactitem will be nice.
\usepackage{paralist}

%% Reference the last page in the page number
%
% NOTE: comment the +LP line and uncomment the -LP line to have page
%       numbers without the ``of ##'' last page reference)
%
% NOTE: uncomment the \pagestyle{empty} line to get rid of all page
%       numbers (make sure includefoot is commented out above)
%
\usepackage{fancyhdr,lastpage}
\pagestyle{fancy}
%\pagestyle{empty}      % Uncomment this to get rid of page numbers
\fancyhf{}\renewcommand{\headrulewidth}{0pt}
\fancyfootoffset{\marginparsep+\marginparwidth}
\newlength{\footpageshift}
\setlength{\footpageshift}
          {0.5\textwidth+0.5\marginparsep+0.5\marginparwidth-2in}
\lfoot{\hspace{\footpageshift}%
       \parbox{4in}{\, \hfill %
                    \arabic{page} of \protect\pageref*{LastPage} % +LP
%                    \arabic{page}                               % -LP
                    \hfill \,}}

% Finally, give us PDF bookmarks
\usepackage{color,hyperref}
\definecolor{darkblue}{rgb}{0.0,0.0,0.3}
\hypersetup{colorlinks,breaklinks,
            linkcolor=darkblue,urlcolor=darkblue,
            anchorcolor=darkblue,citecolor=darkblue}

%%%%%%%%%%%%%%%%%%%%%%%% End Document Setup %%%%%%%%%%%%%%%%%%%%%%%%%%%%


%%%%%%%%%%%%%%%%%%%%%%%%%%% Helper Commands %%%%%%%%%%%%%%%%%%%%%%%%%%%%

% The title (name) with a horizontal rule under it
%
% Usage: \makeheading{name}
%
% Place at top of document. It should be the first thing.
\newcommand{\makeheading}[1]%
        {\hspace*{-\marginparsep minus \marginparwidth}%
         \begin{minipage}[t]{\textwidth+\marginparwidth+\marginparsep}%
                {\large \bfseries #1}\\[-0.15\baselineskip]%
                 \rule{\columnwidth}{1pt}%
         \end{minipage}}

% The section headings
%
% Usage: \section{section name}
%
% Follow this section IMMEDIATELY with the first line of the section
% text. Do not put whitespace in between. That is, do this:
%
%       \section{My Information}
%       Here is my information.
%
% and NOT this:
%
%       \section{My Information}
%
%       Here is my information.
%
% Otherwise the top of the section header will not line up with the top
% of the section. Of course, using a single comment character (%) on
% empty lines allows for the function of the first example with the
% readability of the second example.
\renewcommand{\section}[2]%
        {\pagebreak[2]\vspace{1.3\baselineskip}%
         \phantomsection\addcontentsline{toc}{section}{#1}%
         \hspace{0in}%
         \marginpar{
         \raggedright \scshape #1}#2}

% An itemize-style list with lots of space between items
\newenvironment{outerlist}[1][\enskip\textbullet]%
        {\begin{itemize}[#1]}{\end{itemize}%
         \vspace{-.6\baselineskip}}

% An environment IDENTICAL to outerlist that has better pre-list spacing
% when used as the first thing in a \section
\newenvironment{lonelist}[1][\enskip\textbullet]%
        {\vspace{-\baselineskip}\begin{list}{#1}{%
        \setlength{\partopsep}{0pt}%
        \setlength{\topsep}{0pt}}}
        {\end{list}\vspace{-.6\baselineskip}}

% An itemize-style list with little space between items
\newenvironment{innerlist}[1][\enskip\textbullet]%
        {\begin{compactitem}[#1]}{\end{compactitem}}
        
% An itemize-style list with little space between items
\newenvironment{innerinnerlist}[1][\enskip$\circ$]%
        {\begin{compactitem}[#1]}{\end{compactitem}}

% An environment IDENTICAL to innerlist that has better pre-list spacing
% when used as the first thing in a \section
\newenvironment{loneinnerlist}[1][\enskip\textbullet]%
        {\vspace{-\baselineskip}\begin{compactitem}[#1]}
        {\end{compactitem}\vspace{-.6\baselineskip}}

% To add some paragraph space between lines.
% This also tells LaTeX to preferably break a page on one of these gaps
% if there is a needed pagebreak nearby.
\newcommand{\blankline}{\quad\pagebreak[2]}

% Uses hyperref to link DOI
\newcommand\doilink[1]{\href{http://dx.doi.org/#1}{#1}}
\newcommand\doi[1]{doi:\doilink{#1}}


%%%%%%%%%%%%%%%%%%%%%%%% End Helper Commands %%%%%%%%%%%%%%%%%%%%%%%%%%%

%%%%%%%%%%%%%%%%%%%%%%%%% Begin CV Document %%%%%%%%%%%%%%%%%%%%%%%%%%%%

\begin{document}
\makeheading{Barend~J.~Leonard \hfill Curriculum Vitae}

\section{Contact Information}
%
% NOTE: Mind where the & separators and \\ breaks are in the following
%       table.
%
% ALSO: \rcollength is the width of the right column of the table
%       (adjust it to your liking; default is 1.85in).
%
\newlength{\rcollength}\setlength{\rcollength}{1.85in}%
%
\begin{tabular}[t]{@{}p{\textwidth-\rcollength}p{\rcollength}}
\href{http://www.cs.up.ac.za/}%
     {Department of Computer Science} & \textit{Cell:} (082) 744-6701 \\
\href{http://www.up.ac.za/}{The University of Pretoria}
                           & \textit{Work:} (012) 420-5242 \\
Lynwood Rd.                &  \textit{E-mail:}\\
Hatfield, Pretoria, South Africa    & \href{mailto:bleonard@cs.up.ac.za}{bleonard@cs.up.ac.za}\\ 
\end{tabular}

\section{Citizenship}
%
South Africa

\section{Background}
%
I am a computer scientist with a deep understanding and working knowledge of a wide range of computational intelligence techniques. I am especially familiar with the application of swarm intelligence to machine learning. I have worked under the supervision of Prof. Engelbrecht at the University of Pretoria for the past four years. During this time I was required to implement and test many existing and newly developed computational intelligence algorithms. In addition, I was the lead author of a number of publications that were presented at leading international conferences on computational intelligence. Prof. Engelbrecht is a world leader in the field of computational swarm intelligence and is currently the only A-rated computer scientist in South Africa. I am currently registered for an M.Sc. in computational intelligence under his supervision.

~~~~I enjoy analysing and critiquing existing algorithms and solving problems in the context of computational intelligence. I have excellent report writing skills and a broad range of scientific interests. Finding an industry position where I can apply my knowledge to real-world problems would be ideal for me.

\section{Basic\\Education}
%
\href{}{\textbf{Ho\"erskool Garsfontein}}, Pretoria, South Africa
\begin{outerlist}
\item[] \textbf{\textit{Matric}} \hfill \textbf{2004 - Passed with endorsement}
        \begin{innerlist}
        \item Mathematics
        \item Science
        \item Computer Studies
        \item Afrikaans 1st language
        \item English 2nd language
        \item Accounting
        \end{innerlist}
\end{outerlist}

\section{Higher Education}
%
\href{http://www.up.ac.za/}{\textbf{The University of Pretoria}},
Pretoria, South Africa
\begin{outerlist}
\item[] \textbf{\textit{M.Sc. (\href{http://www.cs.up.ac.za/}{Computer Science})}} \hfill \textbf{Expected graduation date: November 2015}
        \begin{innerlist}
        \item \emph{Specialisation:} Computational intelligence
        \item \emph{Thesis title:} Angle Modulated Particle Swarm Optimisation.
        \begin{innerinnerlist}
        		\item My research is focussed on the application of particle swarm optimisation to discrete optimisation problems. The work included a thorough theoretical and empirical analysis of an existing technique, known as \emph{angle modulation}. A number of flaws in the technique were identified and solutions were developed to circumvent those problems. The final stage of the research is to conduct an empirical study to show that the newly developed methods outperform existing techniques.
        \end{innerinnerlist}
        \item \emph{Derived publications:} 1 published, 1 under review (see Publications section for details.)
        \item \emph{Supervisor:} \href{http://www.cs.up.ac.za/engel/}{Prof. A.P. Engelbrecht}\\
        \end{innerlist}
\end{outerlist}

\pagebreak
\href{http://www.up.ac.za/}{\textbf{The University of Pretoria}},
Pretoria, South Africa
\begin{outerlist}
\item[] \textbf{\textit{B.Sc. (Hons) (\href{http://www.cs.up.ac.za/}{Computer Science})}} \hfill \textbf{2011}
        \begin{innerlist}
        \item \emph{Areas of focus:} Computational intelligence, data mining, computer security, formal methods for algorithm construction and analysis, computer networks theory, research methods.
        \item \emph{Research project title:} Empirical Analysis of Heterogeneous Particle Swarms in Dynamic Environments.
        \item \emph{Supervisor:} \href{http://www.cs.up.ac.za/engel/}{Prof. A.P. Engelbrecht}\\
        \end{innerlist}
\end{outerlist}

\href{http://www.up.ac.za/}{\textbf{The University of Pretoria}},
Pretoria, South Africa
\begin{outerlist}
\item[] \textbf{\textit{B.Sc. (\href{http://www.cs.up.ac.za/}{Computer Science})}} \hfill \textbf{2010}
        \begin{innerlist}
        \item \emph{Areas of focus:} Computational intelligence, programming language theory, computer networks, databases, discrete data structures, software engineering principles \& practices.
        \item \emph{Programming languages:} Java, C, C++, JavaScript, PHP and others.\\
        \end{innerlist}
\end{outerlist}

\section{Certificates}
%
\href{https://www.duke.edu/}{\textbf{Duke University}},
North Carolina, United States of America
\begin{outerlist}
\item[] \textbf{\textit{Verified certificate: Introduction to genetics and evolution}} \hfill \textbf{2015}
        \begin{innerlist}
        \item \emph{Subject material:} A whirlwind introduction to evolution and genetics, from basic principles to current applications, including how disease genes are mapped, areas of research in evolutionary genetics, and how evolutionary concepts are leveraged to aid humanity.
        \item Completion of this certificate gave me a deeper understanding of the evolutionary concepts that are used in the design of evolutionary algorithms. (Evolutionary algorithms is a field of specialisation in computational intelligence and machine learning.)
        %\item \emph{Link to verified certificate:}
        \end{innerlist}
\end{outerlist}

\section{Work Experience}
%
\href{http://www.up.ac.za}{\textbf{The University of Pretoria}}, Pretoria, South Africa
\begin{outerlist}
\item[] \textbf{\textit{Programmer and Researcher}} \hfill \textbf{January 2012 to  present}
        \begin{innerlist}
        \item Contribution of code to the \href{http://www.cilib.net}{CILib} open-source computational intelligence library.
        \item Design, implementation and maintenance of a platform-independent cluster for the execution of large-scale computational intelligence experiments\\(including both server-side and client side programming and web design).
        \item Configuration and maintenance of Unix servers and other computing infrastructure.
        \item Presenting international workshops and tutorials on the use of CILib when conducting computational intelligence research.\\
        \end{innerlist}
\end{outerlist}
\textit{Reference at the Department of Computer Science:} \hfill Prof. A.P. Engelbrecht \\ .\hfill (engel@cs.up.ac.za) \\

\pagebreak
\href{http://www.up.ac.za}{\textbf{The University of Pretoria}}, Pretoria, South Africa
\begin{outerlist}
\item[] \textbf{\textit{Research Assistant}} \hfill \textbf{March 2010 to  December 2011}
        \begin{innerlist}
        \item Implementation of modern computational intelligence algorithms for research purposes.
        \item Contribution of code to the \href{http://www.cilib.net}{CILib} open-source computational intelligence library.\\
        \end{innerlist}
\end{outerlist}
\textit{Reference at the Department of Computer Science:} \hfill Prof. A.P. Engelbrecht \\ .\hfill (engel@cs.up.ac.za) \\

\href{http://www.meraka.org.za}{\textbf{Meraka Institute}}, CSIR, Pretoria, South Africa
\begin{outerlist}
\item[] \textbf{\textit{Software Engineer}} \hfill \textbf{March 2009 to  November 2009}
        \begin{innerlist}
        \item Designed and implemented a multi-user text based game engine (MUD) for Dr Math on MXit.
        \item The project was done in partial fulfilment of a \textit{B.Sc. (Computer Science)} degree at the University of Pretoria.\\
        \end{innerlist}
\end{outerlist}
\textit{Reference at CSIR Meraka Institute:} \hfill Laurie Butgereit (lbutgereit@csir.co.za) \\

%\pagebreak

\href{http://www.up.ac.za}{\textbf{The University of Pretoria}}, Pretoria, South Africa
\begin{outerlist}
\item[] \textbf{\textit{Laboratory Assistant}} \hfill \textbf{March 2008 to  February 2010}
        \begin{innerlist}
        \item Maintenance of computers and printers in the laboratory.
        \item Providing assistance to students in any of the following areas:
        \begin{innerinnerlist}
        \item Logging onto campus network using Novell.
        \item Use of word processors, spreadsheet applications and presentation design software (in both Windows and Linux environments).
        \item Sending of e-mails using the university mail server, or third party mail servers such as GMail.
        \item Provide support to the best of my ability on any other issues students may experience in the laboratory. \\
        \end{innerinnerlist}
        \end{innerlist}
\end{outerlist}
\textit{Reference at the UP Open Labs:} \hfill Retief Joyce (retief.joyce@up.ac.za)


\section{Teaching Experience}
\href{http://www.up.ac.za}{\textbf{The University of Pretoria}}, Pretoria, South Africa
\begin{outerlist}
\item[] \textbf{\textit{Third Year and Honours Teaching}} \hfill \textbf{2012 to present}
        \begin{innerlist}
        \item Lectures on the use and development of the CiLib computational intelligence library. \\
        \end{innerlist}
\end{outerlist}

\href{http://www.ufpe.br/english/}{\textbf{University of Pernambuco}}, Recife, Brazil
\begin{outerlist}
\item[] \textbf{\textit{Third Year and Honours Teaching}} \hfill \textbf{2012}
        \begin{innerlist}
        \item Presented a series of tutorials on the use and development of the CiLib open source computational intelligence library. \\
        \end{innerlist}
\end{outerlist}

\href{http://brocku.ca}{\textbf{Brock University}}, Toronto, Canada
\begin{outerlist}
\item[] \textbf{\textit{Honours Teaching}} \hfill \textbf{2011}
        \begin{innerlist}
        \item Presented a series of tutorials on the use and development of the CiLib open source computational intelligence library. \\
        \end{innerlist}
\end{outerlist}

%\pagebreak

\section{Technical\\ Skills}
%
\textbf{Programming Languages and Tools:} Java, C, C++, JavaScript, PHP, Python, UNIX
        shell scripting, git, maven, sbt, Netbeans, Eclipse and others.

\blankline

\textbf{Databases:} SQL, mySQL, MongoDB.

\blankline

\textbf{Applications:} \LaTeX{}, B\textsc{ib}\TeX{}, Microsoft Office, Libre Office, and other common productivity packages for Windows and Linux platforms.

\blankline

\textbf{Operating Systems:} Linux and other UNIX variants, Microsoft Windows XP, Windows Vista / Windows 7 / Windows 8 / Windows 10.

\blankline

\textbf{Networking:} Extensive experience in setting up and maintaining wired and wireless networks; Experience programming CISCO routers via telnet in WAN environments; Experience installing and configuring Mikrotik routers via WinBox in both LAN and WAN environments.

\blankline

\textbf{Hardware:} Solid knowledge of computer components and experience in building a variety of computer systems.

\section{Publications} \begin{bibsection}
	\item Leonard, B. and Engelbrecht, Critical Considerations on Angle Modulated Particle Swarm Optimisers. \emph{Swarm Intelligence}, \emph{[under review]}.
	\item Leonard, B. and Engelbrecht, Angle Modulated Particle Swarm Variants. \emph{ANTS 2014}, 2014.
    \item Leonard, B. and Engelbrecht, A. On the optimality of Particle Swarm Optimizers in Dynamic Environments. \emph{2013 Congres on Evolutionary Computation}, 2013.
	\item Leonard, B. and Engelbrecht, A. Scalability study of Particle Swarms in Dynamic Environments. \emph{ANTS 2012}, 2012.
    \item Leonard, B. and Engelbrecht, A. and Van Wyk, A. Heterogeneous particle swarms in dynamic environments. \emph{2011 IEEE Symposium on Swarm Intelligence}, 2011.
    \item Butgereit, L. and Leonard, B. {\it et al}. Dr Math gets MUDDY: the ``dirt" on how to attract teenagers to mathematics and science by using multi-user dungeon games over Mxit on cell phones. \emph{IST Africa 2010}, 2010.
\end{bibsection}

\end{document}

%%%%%%%%%%%%%%%%%%%%%%%%%% End CV Document %%%%%%%%%%%%%%%%%%%%%%%%%%%%%
